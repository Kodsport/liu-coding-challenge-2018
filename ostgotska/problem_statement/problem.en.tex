\problemname{Östgötska}
Anders talks in the Swedish dialect of \emph{östgötska}.
Unfortunately, this makes it somewhat hard to take him seriously.

He is currently applying for a job in the Swedish capital of Stockholm.
To increase his chances of passing interviews, he wishes to practice the standard Swedish dialect,
\emph{rikssvenska}.
To help him with this, he wants you to write a program that can determine whenever he accidentally reverts to speaking östgötska.

A simple way of determining if a sentence is written in östgötska is if at least $40\%$ of the words in the sentence contains the letter \texttt{ä}.
For simplicity, this is encoded as the letter combination \texttt{ae} (meaning any appearance of the substring \texttt{ae} is to be regarded as an occurrence of the letter \texttt{ä}).

\section*{Input}
The first and only line of input contains a sequence of space-separated words.
Each word consists only of letters \texttt{a-z}.
There are at most $100$ words, and each word contains at most $10$ letters.

\section*{Output}
Output ``\texttt{dae ae ju traeligt va}'' if the input sentence is in östgötska, otherwise output ``\texttt{haer talar vi rikssvenska}''.
